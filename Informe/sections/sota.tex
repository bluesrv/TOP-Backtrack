\section{Estado del Arte}\label{sota}
En la literatura, la primera aparici\'on del problema TOP aparece sin llamarse como tal en el art\'iculo de Butt y Cavalier~\cite{BUTT1994101} sino que bajo el nombre de Multiples Tour de M\'axima Colecci\'on. El t\'ermino de TOP es introducido por primera vez en el art\'iculo de Chao et al.~\cite{TOPresume} donde por primera vez se le realiza el ligado de este problema al deporte de Orienteering y por lo tanto como un derivado de OP. Las instanciaciones utilizadas para el testeo en el art\'iculo de Chao et al.~\cite{TOPresume} son utilizadas hasta hoy en d\'ia para realizar el testeo de nuevos acercamientos al problema.

Para todo problema de este tipo existen dos formas de buscar soluciones, a trav\'es de m\'etodos exactos y de m\'etodos inexactos, donde estos \'ultimos contemplan heur\'isticas y metaheur\'isticas. Como diferencia entre ambos se puede destacar entre todo las limitaciones que tienen, ya que un m\'etodo exacto entregar\'a \'optimos globales pero a una cantidad discreta de instanciaciones del problema en tiempo razonable, mientras que un m\'etodo inexacto puede ser mas r\'apido y abordar mas instancias pero lo hace a costo de que el \'optimo que haya encontrado sea local y no global. Es por esto que se revisar\'a el estado del arte para ambos m\'etodos y los trabajos previos para las implementaciones mas eficientes actuales.

\subsection{Metodolog\'ia Exacta}

En acercamientos exactos, es com\'un ver que para obtener mejores resultados no tan s\'olo se buscan varaiciones de algoritmos de b\'usqueda combinatorial, sino que adem\'as en la mayor\'ia de las ocaciones se acompa\~na con una reformulaci\'on del modelo matem\'atico que describe el problema. El primer acercamiento con metodolog\'ia exacta del cual se logr\'o obtener informaci\'on fue del art\'iculo de Butt y Ryan~\cite{BUTT1999427} donde proponen utilizar el algoritmo de generaci\'on de columnas para obtener soluciones globales \'optimas para el problema. Sin embargo, este m\'etodo s\'olo puede resolver una cantidad muy limitada de instancias del problema en un tiempo razonable. Es por esto que en su trabajo, Boussier, Feillet y Gendreau~\cite{exactAlgorithm} utilizan una metodolog\'ia basada en el algoritmo de ramificaci\'on y precio, permitiendo resolver instancias mas grandes con hasta 100 v\'ertices. En el trabajo de Poggi, Viana y Uchoa~\cite{poggi_et_al:OASIcs:2010:2756} logran generar un m\'etodo competitivo con el anterior reformulando el problema y luego aplicando algoritmos de ramificaci\'on corte y precio. Lo m\'as actual es el trabajo de Bianchessi, Mansini y Garzia Speranza~\cite{doi:10.1111/itor.12422} donde no s\'olamente resuelven de forma \'optima 10 instancias m\'as que la mejor anterior, utilizando su implementaci\'on de hebra simple y 26 para la implementaci\'on de multi-hebras, sino que adem\'as resuelve 24 instancias previamente sin resolver. Para esto, lo que hacen es reformular el problema de forma de matem\'aticamente modelarlo como un problema de dos \'indices y cantidad polin\'omica de variables, las cuales se agrupan en variables binarias, enteras y continuas; donde una soluci\'on viable se representa con un arreglo binario de tama\~no constante para indicar si el nodo es visitado o no, una matriz cuadrada binaria que representa los arcos y si estos son utilizados o no, y una matriz con valores continuos que indican el momento de llegada al pasar por ese arco; para luego introducir en el modelo restricciones de conectividad, y as\'i resolvi\'endolo utilizando algoritmos de ramificaci\'on y corte. Ya que la cantidad de restricciones de conectividad crece exponencialmente con respecto a la cantidad de puntos de control, estos son agregados din\'amicamente al \'arbol de ramificar-y-enlazar. Tanto el modelo originalmente utilizado como la refomulaci\'on de este \'ultimo art\'iculo se encontrar\'an presentes en la Secci\'on \ref{model}.

\subsection{Metodolog\'ia Inexacta}

Debido a la naturaleza NP-dif\'icil de este problema, la mayor\'ia de los esfuerzos de investigaci\'on sobre el se concentran en el uso de heur\'isticas y metaeur\'isticas, raz\'on por la cual se encuentra una mayor cantidad de art\'iculos con variedad de alogoritmos utilizados en el \'ambito del acercamiento inexacto. En su trabajo, Butt y Cavalier~\cite{BUTT1994101} proponen un procedimiento greedy, cuando el problema a\'un no se conoc\'ia como TOP. A su vez Chao et al.~\cite{TOPresume} cuando propuso el nombre de TOP para este problema, propuso un m\'etodo de cinco pasos y una heur\'istica basada en el algoritmo estoc\'astico de Tsiligirides~\cite{Tsiligirides1984}. B\'usqueda tab\'u fue propuesta por Tang y Miller-Hooks~\cite{TANG20051379}, mientras que Archetti, Hertz y Speranza~\cite{Archetti2007} proponen un algoritmo de b\'usqueda de dos tab\'ues y un algoritmo de b\'usqueda con vecindario cambiante. Ke, Archetti y Feng~\cite{KE2008648} fueron los primeros en utilizar algoritmos de metaheur\'isticas basados en la metodolog\'ia de enjambre (swarming) donde a trav\'es de la t\'ecnica de Ant Colony Optimization~\cite{dorigo2011ant} (ACO) se aplica en base a cuatro m\'etodos por sobre el problema de TOP, un m\'etodo secuencial, un m\'etodo concurrente dividido en determinista y aleatorio, y un m\'etodo simultaneo. En su trabajo, el cual ellos denominan ACO-TOP, obtienen resultados que a la fecha compet\'ian eficiente y efectivamente contra los que ya exist\'ian, donde el m\'etodo secuencial destacaba por obtener la mejor calidad de resoluci\'on en menos de un minuto por cada instancia. Posterior a ACO-TOP, Vansteenwegen et al.~\cite{VANSTEENWEGEN2009118} proponen un algoritmo de b\'usqueda local guiada y un algoritmo de vecindario con variable sesgada. Bouly et al.~\cite{Bouly2010} luego proponen un algoritmo mem\'etico, que es b\'asicamente un algoritmo gen\'etico con m\'etodos h\'ibridos dedicados espec\'ificamente a TOP, un procedimiento dividido \'optimo para la evaluaci\'on cromos\'omica, y t\'ecnicas de b\'usqueda local para la mutaci\'on. Dang, Guidabj y Moukrim~\cite{dang2013effective} proponen un algoritmo inspirado en enjambramiento de particulas logrando reducir el error relativo con respecto a las evaluaciones con metodolog\'ias exactas de un 0.0005\%, por lo cual es considerado como uno de los algoritmos mas efectivos a la fecha, a pesar de no ser tan eficiente en t\'erminos de tiempo de ejecuci\'on. Adem\'as, introdujeron un nuevo conjunto de instancias m\'as grande para benchmarking. Recientemente Ke, Zhai, Li y Chan~\cite{KE2016155} introdujeron una nueva metaheur\'istica basada en la metaheuristica de poblaci\'on, aplic\'andola por primera vez sobre un problema TOP, por lo que se le conoce como el acercamiento mas grande a resolver TOP, utilizando lo que ellos denominaron como un Algoritmo de Imitaci\'on de Pareto (Pareto mimic algorithm PMA) donde utilizan un nuevo operador, un operador de imitaci\'on, para generar una nueva soluci\'on representada como un arreglo de tama\~no m con m caminos, al simplemente imitar una soluci\'on titular. Igualmente adopta un operador swallow, de forma de tragar, o insertar, un nodo inviable y luego reparar la soluci\'on inviable resultante.~\cite{GUNAWAN2016315} Finalmente, Ke et al.~\cite{KE2016155} lograron obtener diez mejoras a las soluciones existentes ya conocidas para las instancias de benchmark de Chao et al.~\cite{TOPresume} y de Dang et al.~\cite{dang2013effective}.

\subsection{Orientaci\'on de los Esfuerzos Actuales}

Se observa cl\'aramente una tendencia a reformular matem\'aticamente el modelo o las heur\'isticas con las que se trabaja el problema, la tendencia actual se orienta a tratar de abordar el problema desde nuevas perspectivas en vez de tratar de reimplementar m\'etodos a modelos ya existentes de alguna forma \'optima de acuerdo a lo que el poder computacional o de estructuraci\'on de c\'odigo permita.