\section{Definici\'on del Problema}\label{define}
Aunque en la Secci\'on \ref{intro} ya se dio una idea de como se puede observar un problema de TOP haci\'endo la analog\'ia con el deporte, no todo problema de este tipo estar\'a necesariamente ligado al deporte. Es por esto que es necesario dar una definici\'on mas general, en un lenguaje natural, como funciona un problema de Team Orienteering.

De acuerdo a como Boussier~\cite{exactAlgorithm} lo define, un problema de Team Orienteering puede ser visto como un grafo constituido de los puntos de control, que ser\'ian v\'ertices etiquetados con el valor del premio por visitarlos, y los arcos como las rutas entre ellos, cuya etiqueta ser\'ia la distancia entre los nodos que conecta. El objetivo de TOP es encontrar un conjunto de rutas (una ruta por cada participante) sujetas a un tiempo determinado para recorrerlas, recorriendo cada nodo intermedio a los mas una vez y cuyo premio acumulado por visitarlos sea m\'aximo. Generalmente, el problema se simplifica asumiendo una velocidad de recorrido de 1 unidad de distancia por 1 unidad de tiempo, por lo que el tiempo m\'aximo pasa a ser una distancia m\'axima recorrida, haciendo que la suma total de las distancias de cada ruta sea la distancia recorrida y que sea eso lo que se limite en vez del tiempo.

As\'i como se realiza la analog\'ia de TOP con el deporte al aire libre, igualmente se le asocia el problema de enrutamiento de veh\'iculos para la visita de clientes y con distancia m\'axima de recorrido, la cual constituye un gran esfuerzo de log\'istica en muchos contextos de la industria actual, como por ejemplo la visita de t\'ecnicos a clientela, planificaci\'on de viajes tur\'isticos, etc.

De igual forma que TOP proviene de OP, muchos problemas han sido derivados de TOP, que a grandes rasgos constituyen variaciones donde se agregan restricciones a las dimensiones existentes en el problema. Por ejemplo, existe Team Orienteering Problem With Time Window, abreviado como TOPTW (Igualmente conocido como Selective Vehicle Routing Problem with Time Windows, SVRPTW~\cite{gueguen1999methodes}), donde se tiene un cierto margen de tiempo para visitar un nodo, lo cual aplica mas realistamente a un itinerario de un paquete tur\'istico o a un itinerario de actividades en un evento. Tambi\'en, otra variante del problema es Capacitated Team Orienteering Problem (CTOP~\cite{doi:10.1057/palgrave.jors.2602603}) cuya primicia recae en que los clientes adem\'as poseen una demanda que cumplir en cada visita, mientras que los veh\'iculos poseen una capacidad m\'axima de carga, lo cual se puede reflejar en la industria con un servicio de distribuci\'on. Finalmente, cabe destacar que as\'i como existen variaciones a TOP, existen variaciones como especificaciones de variantes del mismo, o mezcla de variaciones que se convierten en una variaci\'on de por s\'i, como es el caso del  Multiconstraint Team Orienteering Problem with Multiple Time Windows (MC-TOP-MTW), por lo que existen varias formas de especializar el problema y varios son arduamente trabajados debido a su aplicaci\'on en la log\'istica de la industria actual.