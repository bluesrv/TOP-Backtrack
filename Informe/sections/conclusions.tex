\section{Conclusiones}\label{conclusions}
Respecto a aproximaciones anteriores, no tan solamente la metodolog\'ia utilizada juega un rol importante en la calidad de las soluciones y en el tiempo en que estas se obtienen, sino que adem\'as es relevante las representaciones que se utilizan para esto. Es as\'i como en Bianchessi et al.\cite{doi:10.1111/itor.12422} notamos importantes resultados nuevos en cuanto a benchmarking y a la obtensi\'on de nuevos \'optimos para considerar globales al cambiar el modelo matem\'atico del algoritmo exacto para que este funcione con variables de a lo sumo dos indices de dimensi\'on. Mientras que en Ke et al.\cite{KE2016155} notamos que en busquedas locales se logran mejores resultados al cambiar las operaciones internas del algoritmo al momento de elegir nuevas posibles soluciones para diversificar e intensificar, lo cual llev\'o a la creaci\'on de una nueva metaheur\'istica. Aunque la mayor\'ia de los papers visitados en este documento resuelven instancias del problema TOP, ya que as\'i fueron buscados los documentos, no todos logran obtener resultados favorables en cuantro a encontrar nuevos resultados globales o tiempos de ejecuci\'on notoriamente mas r\'apidos, en general muestran nuevas aproximaciones competitivas con respecto a los mejores, es tan s\'olo en los \'ultimos dos papers mencionados recientemente donde se logran resultados significativos en a\~nos, debido a, como ya se mencion\'o, la utilizaci\'on de nuevas representaciones, modelos, y metaheur\'isticas.

Respecto a este acercamiento, la elecci\'on de BackTrack con GBJ termin\'o llevando a la implementaci\'on de un BackTrack com\'un y corriente, debido a que como el problema posee un grafo de restricciones totalmente conexo, el salto al paso anterior parece tan bueno como cualquier otro para que cumpla con el requisito de ser un salto GBJ. Quiz\'as CBJ podr\'ia haber resultado mas eficiente, debido a que por la naturaleza de los caminos, al momento del algoritmo revisar las conexiones finales habr\'ia solamente deshecho los \'ultimos v\'ertices que visitan las rutas conflictivas, lo cual podr\'ia haber reducido tiempo de ejecuci\'on de manera considerable, aunque a medida que se aumenta el tiempo m\'aximo de cada ruta ambas heur\'isticas de salto habr\'ian convergido igualmente al mismo tiempo de ejecuci\'on.

Aunque el algoritmo propuesto entrega soluciones a la mayor\'ia de los casos de prueba existente en la literatura en tiempos que bordean las horas, no est\'a entregando resultados similares a los que de referencia debiese dar, es por esto que aunque se considera un buen acercamiento para obtener resultados globales, no est\'a bien logrado. A pesar de esto, se considera dentro de las posibles implementaciones de BackTrack para este problema, como una implementaci\'on competitiva y hasta mas eficiente que la media al ocupar depresentaciones de a lo mas dos \'indices, inspir\'andose en el modelo propuesto por Bianchessi et al.\cite{doi:10.1111/itor.12422}.

Para mejorar este algoritmo, una buena forma ser\'ia arregl\'andolo para que entregue las soluciones globales en las intancias que no est\'an correspondiendo, lo cual se deduce es debido a un problema puntual de implementaci\'on y no del algoritmo en si, ya que revisi\'on tras revisi\'on se ha verificado que la recursi\'on en su para recorrer el espacio de b\'usqueda est\'a bien hecho. Respecto al tiempo de ejecuci\'on, no es posible al momento de entregar este informe determinar alguna forma para mejorarlo.