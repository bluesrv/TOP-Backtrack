\section{Introducci\'on}\label{intro}
Orienteering es un deporte al aire libre que consiste en que un participante, desde de un punto de partida, debe visitar distintos puntos de control hasta terminar en un punto de llegada dentro de un margen delimitado de tiempo. Cada punto de control tiene un puntaje asociado el cual se le otorgar\'a al participante s\'olo la primera vez que pase por all\'i, por lo que el ganador ser\'a aquel que haya juntado la mayor cantidad de puntos tras haber llegado a la meta dentro del tiempo disponible, implicando que no necesariamente visitar\'a la mayor cantidad de puntos respecto a otros participantes, es por esto que responder cual es la ruta que se debe tomar para ganar no es trivial y se puede modelar como un problema de optimizaci\'on.

El problema de Orienteering para un s\'olo participante, que es el que se acaba de describir, envuelve mas capas de complejidad cuando uno cambia el enfoque de un solo participante a un equipo participante con dos o mas integrantes. En este caso el problema pasa a ser intuitivamente conocido como uno de Team Orienteering Problem, o TOP para abreviar. Ya que el problema no s\'olo se trata de encontrar un camino que entregue la mayor cantidad de puntos, sino que varios caminos para que puedan ser recorridos en paralelo por los integrantes del grupo dentro del margen de tiempo y que en conjunto me den el mayor puntaje posible. Considerando que el problema de Orienteering para un solo participante es NP-dif\'icil~\cite{opComplex}, TOP tiene al menos la misma complejidad que este al ser un derivado de el~\cite{TOPresume}.

Debido a la naturaleza de este problema, es de esperarse que multiples cientistas de la computaci\'on en el \'area de la investigaci\'on operacional hayan intentado abordarlo a trav\'es del uso de distintas t\'ecnicas para encontrar soluciones que satisfagan lo planteado en un problema TOP de forma eficiente para una generalidad de instanciaciones, esto debido a la fuerte similitud que este problema tiene con respecto a operaciones en la industria actual, a ra\'iz de los cuales se producen variantes de TOP las cuales ser\'an discutidos mas adelante.

Finalmente es por todo lo ya explicado que se redacta este documento, con la finalidad de dar revisi\'on a estos diversos acercamientos al problema TOP, las mejoras y variedades entre cada uno de ellos y a las variantes que han surgido del mismo. Adem\'as igualmente se propone una soluci\'on al problema utilizando una t\'ecnica de Backtracking con Graph Back Jump (GBJ), se describe su implementaci\'on en base a sus representaciones y se comparan sus resultados con los actualmente existentes. En lo que queda de este art\'iculo, se describe el problema en detalle denotando sus variables, restricciones y funci\'on objetivo en la Secci\'on \ref{define}. Luego, en la Secci\'on \ref{sota} se presentan diversos acercamientos al problema resaltando su instante conol\'ogico y metodolog\'ia aplicada. En la Secci\'on \ref{model} se entrega un modelo matem\'atico del problema. Posteriormente en la Secci\'on \ref{represent} se describe la representaci\'on utilizada en la soluci\'on propuesta, en la Secci\'on \ref{descript} se describe el algoritmo utilizado y en las Secciones \ref{experiments} y \ref{results} se muestran los experimentos para las intancias que se resolvieron utilizando el c\'odigo de la soluci\'on propuesta y sus respectivos resultados. Finalmente, en la Secci\'on \ref{conclusions} se concluye de acuerdo a la revisi\'on realizada y a la soluci\'on propuesta respecto de los resultados obtenidos y posibles mejoras al problema.