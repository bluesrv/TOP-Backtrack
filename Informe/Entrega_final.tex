\documentclass[letter, 10pt]{article}
%%\usepackage[latin1]{inputenc}
\usepackage[spanish]{babel}
\usepackage{amsfonts}
\usepackage{amsmath}
\usepackage[dvips]{graphicx}
\usepackage{url}
\usepackage[top=3cm,bottom=3cm,left=3.5cm,right=3.5cm,footskip=1.5cm,headheight=1.5cm,headsep=.5cm,textheight=3cm]{geometry}
\usepackage[]{algorithm2e}
\usepackage{tabularx}
\usepackage{multirow}
\usepackage{float}

% Footnote size was too big
\usepackage[bottom]{footmisc}

% References reformatting for personal acomodations
\usepackage[numbib]{tocbibind}
\renewcommand{\refname}{Referencias}
\renewcommand{\abstractname}{Resumen}

% Keywords command
\providecommand{\keywords}[1]
{
  \small	
  \textbf{\textit{Keywords:}} #1
}

\begin{document}
\title{Inteligencia Artificial \\ \begin{Large}Informe Final: Team Orienteering Problem\end{Large}}
\author{Sebasti\'an Ignacio Ram\'irez Vidal}
\date{\today}
\maketitle


%--------------------No borrar esta secci\'on--------------------------------%
\section*{Evaluaci\'on}

\begin{tabular}{ll}
Mejoras 1ra Entrega (10 \%): &  \underline{\hspace{2cm}}\\
C\'odigo Fuente (10 \%): &  \underline{\hspace{2cm}}\\
Representaci\'on (15 \%):  & \underline{\hspace{2cm}} \\
Descripci\'on del algoritmo (20 \%):  & \underline{\hspace{2cm}} \\
Experimentos (10 \%):  & \underline{\hspace{2cm}} \\
Resultados (10 \%):  & \underline{\hspace{2cm}} \\
Conclusiones (20 \%): &  \underline{\hspace{2cm}}\\
Bibliograf\'ia (5 \%): & \underline{\hspace{2cm}}\\
 &  \\
\textbf{Nota Final (100)}:   & \underline{\hspace{2cm}}
\end{tabular}
%---------------------------------------------------------------------------%

\vspace{2cm}


\begin{abstract}
Un problema com\'un en la industria es la b\'usqueda de rutas \'optimas para realizar servicios que requieren el despliegue de m\'ultiples unidades para recubrir puntos de valor en un horario acotado. Este problema, generalizado, se le conoce como Team Orienteering Problem y en este art\'iculo se proceder\'a a revisar distintos acercamientos a este problema a trav\'es del tiempo con el objetivo de revisar las soluciones, o aproximaciones a soluciones, de cada uno comparando su eficiencia tanto en calidad de resultado como tiempo que demora en obtenerlo, para finalmente proponer una soluci\'on utilizando un algoritmo de Backtracking con GBJ.\\
\keywords{Team Orienteering Problem, M\'etodo exacto, M\'etodo inexacto, Metaheur\'istica, Backtracking}
\end{abstract}


\section{Introducci\'on}\label{intro}
Orienteering es un deporte al aire libre que consiste en que un participante, desde de un punto de partida, debe visitar distintos puntos de control hasta terminar en un punto de llegada dentro de un margen delimitado de tiempo. Cada punto de control tiene un puntaje asociado el cual se le otorgar\'a al participante s\'olo la primera vez que pase por all\'i, por lo que el ganador ser\'a aquel que haya juntado la mayor cantidad de puntos tras haber llegado a la meta dentro del tiempo disponible, implicando que no necesariamente visitar\'a la mayor cantidad de puntos respecto a otros participantes, es por esto que responder cual es la ruta que se debe tomar para ganar no es trivial y se puede modelar como un problema de optimizaci\'on.

El problema de Orienteering para un s\'olo participante, que es el que se acaba de describir, envuelve mas capas de complejidad cuando uno cambia el enfoque de un solo participante a un equipo participante con dos o mas integrantes. En este caso el problema pasa a ser intuitivamente conocido como uno de Team Orienteering Problem, o TOP para abreviar. Ya que el problema no s\'olo se trata de encontrar un camino que entregue la mayor cantidad de puntos, sino que varios caminos para que puedan ser recorridos en paralelo por los integrantes del grupo dentro del margen de tiempo y que en conjunto me den el mayor puntaje posible. Considerando que el problema de Orienteering para un solo participante es NP-dif\'icil~\cite{opComplex}, TOP tiene al menos la misma complejidad que este al ser un derivado de el~\cite{TOPresume}.

Debido a la naturaleza de este problema, es de esperarse que multiples cientistas de la computaci\'on en el \'area de la investigaci\'on operacional hayan intentado abordarlo a trav\'es del uso de distintas t\'ecnicas para encontrar soluciones que satisfagan lo planteado en un problema TOP de forma eficiente para una generalidad de instanciaciones, esto debido a la fuerte similitud que este problema tiene con respecto a operaciones en la industria actual, a ra\'iz de los cuales se producen variantes de TOP las cuales ser\'an discutidos mas adelante.

Finalmente es por todo lo ya explicado que se redacta este documento, con la finalidad de dar revisi\'on a estos diversos acercamientos al problema TOP, las mejoras y variedades entre cada uno de ellos y a las variantes que han surgido del mismo. Adem\'as igualmente se propone una soluci\'on al problema utilizando una t\'ecnica de Backtracking con Graph Back Jump (GBJ), se describe su implementaci\'on en base a sus representaciones y se comparan sus resultados con los actualmente existentes. En lo que queda de este art\'iculo, se describe el problema en detalle denotando sus variables, restricciones y funci\'on objetivo en la Secci\'on \ref{define}. Luego, en la Secci\'on \ref{sota} se presentan diversos acercamientos al problema resaltando su instante conol\'ogico y metodolog\'ia aplicada. En la Secci\'on \ref{model} se entrega un modelo matem\'atico del problema. Posteriormente en la Secci\'on \ref{represent} se describe la representaci\'on utilizada en la soluci\'on propuesta, en la Secci\'on \ref{descript} se describe el algoritmo utilizado y en las Secciones \ref{experiments} y \ref{results} se muestran los experimentos para las intancias que se resolvieron utilizando el c\'odigo de la soluci\'on propuesta y sus respectivos resultados. Finalmente, en la Secci\'on \ref{conclusions} se concluye de acuerdo a la revisi\'on realizada y a la soluci\'on propuesta respecto de los resultados obtenidos y posibles mejoras al problema.
\section{Definici\'on del Problema}\label{define}
Aunque en la Secci\'on \ref{intro} ya se dio una idea de como se puede observar un problema de TOP haci\'endo la analog\'ia con el deporte, no todo problema de este tipo estar\'a necesariamente ligado al deporte. Es por esto que es necesario dar una definici\'on mas general, en un lenguaje natural, como funciona un problema de Team Orienteering.

De acuerdo a como Boussier~\cite{exactAlgorithm} lo define, un problema de Team Orienteering puede ser visto como un grafo constituido de los puntos de control, que ser\'ian v\'ertices etiquetados con el valor del premio por visitarlos, y los arcos como las rutas entre ellos, cuya etiqueta ser\'ia la distancia entre los nodos que conecta. El objetivo de TOP es encontrar un conjunto de rutas (una ruta por cada participante) sujetas a un tiempo determinado para recorrerlas, recorriendo cada nodo intermedio a los mas una vez y cuyo premio acumulado por visitarlos sea m\'aximo. Generalmente, el problema se simplifica asumiendo una velocidad de recorrido de 1 unidad de distancia por 1 unidad de tiempo, por lo que el tiempo m\'aximo pasa a ser una distancia m\'axima recorrida, haciendo que la suma total de las distancias de cada ruta sea la distancia recorrida y que sea eso lo que se limite en vez del tiempo.

As\'i como se realiza la analog\'ia de TOP con el deporte al aire libre, igualmente se le asocia el problema de enrutamiento de veh\'iculos para la visita de clientes y con distancia m\'axima de recorrido, la cual constituye un gran esfuerzo de log\'istica en muchos contextos de la industria actual, como por ejemplo la visita de t\'ecnicos a clientela, planificaci\'on de viajes tur\'isticos, etc.

De igual forma que TOP proviene de OP, muchos problemas han sido derivados de TOP, que a grandes rasgos constituyen variaciones donde se agregan restricciones a las dimensiones existentes en el problema. Por ejemplo, existe Team Orienteering Problem With Time Window, abreviado como TOPTW (Igualmente conocido como Selective Vehicle Routing Problem with Time Windows, SVRPTW~\cite{gueguen1999methodes}), donde se tiene un cierto margen de tiempo para visitar un nodo, lo cual aplica mas realistamente a un itinerario de un paquete tur\'istico o a un itinerario de actividades en un evento. Tambi\'en, otra variante del problema es Capacitated Team Orienteering Problem (CTOP~\cite{doi:10.1057/palgrave.jors.2602603}) cuya primicia recae en que los clientes adem\'as poseen una demanda que cumplir en cada visita, mientras que los veh\'iculos poseen una capacidad m\'axima de carga, lo cual se puede reflejar en la industria con un servicio de distribuci\'on. Finalmente, cabe destacar que as\'i como existen variaciones a TOP, existen variaciones como especificaciones de variantes del mismo, o mezcla de variaciones que se convierten en una variaci\'on de por s\'i, como es el caso del  Multiconstraint Team Orienteering Problem with Multiple Time Windows (MC-TOP-MTW), por lo que existen varias formas de especializar el problema y varios son arduamente trabajados debido a su aplicaci\'on en la log\'istica de la industria actual.
\section{Estado del Arte}\label{sota}
En la literatura, la primera aparici\'on del problema TOP aparece sin llamarse como tal en el art\'iculo de Butt y Cavalier~\cite{BUTT1994101} sino que bajo el nombre de Multiples Tour de M\'axima Colecci\'on. El t\'ermino de TOP es introducido por primera vez en el art\'iculo de Chao et al.~\cite{TOPresume} donde por primera vez se le realiza el ligado de este problema al deporte de Orienteering y por lo tanto como un derivado de OP. Las instanciaciones utilizadas para el testeo en el art\'iculo de Chao et al.~\cite{TOPresume} son utilizadas hasta hoy en d\'ia para realizar el testeo de nuevos acercamientos al problema.

Para todo problema de este tipo existen dos formas de buscar soluciones, a trav\'es de m\'etodos exactos y de m\'etodos inexactos, donde estos \'ultimos contemplan heur\'isticas y metaheur\'isticas. Como diferencia entre ambos se puede destacar entre todo las limitaciones que tienen, ya que un m\'etodo exacto entregar\'a \'optimos globales pero a una cantidad discreta de instanciaciones del problema en tiempo razonable, mientras que un m\'etodo inexacto puede ser mas r\'apido y abordar mas instancias pero lo hace a costo de que el \'optimo que haya encontrado sea local y no global. Es por esto que se revisar\'a el estado del arte para ambos m\'etodos y los trabajos previos para las implementaciones mas eficientes actuales.

\subsection{Metodolog\'ia Exacta}

En acercamientos exactos, es com\'un ver que para obtener mejores resultados no tan s\'olo se buscan varaiciones de algoritmos de b\'usqueda combinatorial, sino que adem\'as en la mayor\'ia de las ocaciones se acompa\~na con una reformulaci\'on del modelo matem\'atico que describe el problema. El primer acercamiento con metodolog\'ia exacta del cual se logr\'o obtener informaci\'on fue del art\'iculo de Butt y Ryan~\cite{BUTT1999427} donde proponen utilizar el algoritmo de generaci\'on de columnas para obtener soluciones globales \'optimas para el problema. Sin embargo, este m\'etodo s\'olo puede resolver una cantidad muy limitada de instancias del problema en un tiempo razonable. Es por esto que en su trabajo, Boussier, Feillet y Gendreau~\cite{exactAlgorithm} utilizan una metodolog\'ia basada en el algoritmo de ramificaci\'on y precio, permitiendo resolver instancias mas grandes con hasta 100 v\'ertices. En el trabajo de Poggi, Viana y Uchoa~\cite{poggi_et_al:OASIcs:2010:2756} logran generar un m\'etodo competitivo con el anterior reformulando el problema y luego aplicando algoritmos de ramificaci\'on corte y precio. Lo m\'as actual es el trabajo de Bianchessi, Mansini y Garzia Speranza~\cite{doi:10.1111/itor.12422} donde no s\'olamente resuelven de forma \'optima 10 instancias m\'as que la mejor anterior, utilizando su implementaci\'on de hebra simple y 26 para la implementaci\'on de multi-hebras, sino que adem\'as resuelve 24 instancias previamente sin resolver. Para esto, lo que hacen es reformular el problema de forma de matem\'aticamente modelarlo como un problema de dos \'indices y cantidad polin\'omica de variables, las cuales se agrupan en variables binarias, enteras y continuas; donde una soluci\'on viable se representa con un arreglo binario de tama\~no constante para indicar si el nodo es visitado o no, una matriz cuadrada binaria que representa los arcos y si estos son utilizados o no, y una matriz con valores continuos que indican el momento de llegada al pasar por ese arco; para luego introducir en el modelo restricciones de conectividad, y as\'i resolvi\'endolo utilizando algoritmos de ramificaci\'on y corte. Ya que la cantidad de restricciones de conectividad crece exponencialmente con respecto a la cantidad de puntos de control, estos son agregados din\'amicamente al \'arbol de ramificar-y-enlazar. Tanto el modelo originalmente utilizado como la refomulaci\'on de este \'ultimo art\'iculo se encontrar\'an presentes en la Secci\'on \ref{model}.

\subsection{Metodolog\'ia Inexacta}

Debido a la naturaleza NP-dif\'icil de este problema, la mayor\'ia de los esfuerzos de investigaci\'on sobre el se concentran en el uso de heur\'isticas y metaeur\'isticas, raz\'on por la cual se encuentra una mayor cantidad de art\'iculos con variedad de alogoritmos utilizados en el \'ambito del acercamiento inexacto. En su trabajo, Butt y Cavalier~\cite{BUTT1994101} proponen un procedimiento greedy, cuando el problema a\'un no se conoc\'ia como TOP. A su vez Chao et al.~\cite{TOPresume} cuando propuso el nombre de TOP para este problema, propuso un m\'etodo de cinco pasos y una heur\'istica basada en el algoritmo estoc\'astico de Tsiligirides~\cite{Tsiligirides1984}. B\'usqueda tab\'u fue propuesta por Tang y Miller-Hooks~\cite{TANG20051379}, mientras que Archetti, Hertz y Speranza~\cite{Archetti2007} proponen un algoritmo de b\'usqueda de dos tab\'ues y un algoritmo de b\'usqueda con vecindario cambiante. Ke, Archetti y Feng~\cite{KE2008648} fueron los primeros en utilizar algoritmos de metaheur\'isticas basados en la metodolog\'ia de enjambre (swarming) donde a trav\'es de la t\'ecnica de Ant Colony Optimization~\cite{dorigo2011ant} (ACO) se aplica en base a cuatro m\'etodos por sobre el problema de TOP, un m\'etodo secuencial, un m\'etodo concurrente dividido en determinista y aleatorio, y un m\'etodo simultaneo. En su trabajo, el cual ellos denominan ACO-TOP, obtienen resultados que a la fecha compet\'ian eficiente y efectivamente contra los que ya exist\'ian, donde el m\'etodo secuencial destacaba por obtener la mejor calidad de resoluci\'on en menos de un minuto por cada instancia. Posterior a ACO-TOP, Vansteenwegen et al.~\cite{VANSTEENWEGEN2009118} proponen un algoritmo de b\'usqueda local guiada y un algoritmo de vecindario con variable sesgada. Bouly et al.~\cite{Bouly2010} luego proponen un algoritmo mem\'etico, que es b\'asicamente un algoritmo gen\'etico con m\'etodos h\'ibridos dedicados espec\'ificamente a TOP, un procedimiento dividido \'optimo para la evaluaci\'on cromos\'omica, y t\'ecnicas de b\'usqueda local para la mutaci\'on. Dang, Guidabj y Moukrim~\cite{dang2013effective} proponen un algoritmo inspirado en enjambramiento de particulas logrando reducir el error relativo con respecto a las evaluaciones con metodolog\'ias exactas de un 0.0005\%, por lo cual es considerado como uno de los algoritmos mas efectivos a la fecha, a pesar de no ser tan eficiente en t\'erminos de tiempo de ejecuci\'on. Adem\'as, introdujeron un nuevo conjunto de instancias m\'as grande para benchmarking. Recientemente Ke, Zhai, Li y Chan~\cite{KE2016155} introdujeron una nueva metaheur\'istica basada en la metaheuristica de poblaci\'on, aplic\'andola por primera vez sobre un problema TOP, por lo que se le conoce como el acercamiento mas grande a resolver TOP, utilizando lo que ellos denominaron como un Algoritmo de Imitaci\'on de Pareto (Pareto mimic algorithm PMA) donde utilizan un nuevo operador, un operador de imitaci\'on, para generar una nueva soluci\'on representada como un arreglo de tama\~no m con m caminos, al simplemente imitar una soluci\'on titular. Igualmente adopta un operador swallow, de forma de tragar, o insertar, un nodo inviable y luego reparar la soluci\'on inviable resultante.~\cite{GUNAWAN2016315} Finalmente, Ke et al.~\cite{KE2016155} lograron obtener diez mejoras a las soluciones existentes ya conocidas para las instancias de benchmark de Chao et al.~\cite{TOPresume} y de Dang et al.~\cite{dang2013effective}.

\subsection{Orientaci\'on de los Esfuerzos Actuales}

Se observa cl\'aramente una tendencia a reformular matem\'aticamente el modelo o las heur\'isticas con las que se trabaja el problema, la tendencia actual se orienta a tratar de abordar el problema desde nuevas perspectivas en vez de tratar de reimplementar m\'etodos a modelos ya existentes de alguna forma \'optima de acuerdo a lo que el poder computacional o de estructuraci\'on de c\'odigo permita.
\section{Modelo Matem\'atico}\label{model}
Como mencionado en la Secci\'on \ref{sota}, existen dos modelos matem\'aticos para este problema dignos de mencionar. El primero corresponder\'ia a la representaci\'on general del problema utilizado en la gran mayor\'ia de los articulos revisados, el cual es extra\'ido del articulo de Ke et al.~\cite{KE2008648}. Mientras que el segundo modelo a presentar corresponde al utilizado por Bianchessi et al.~\cite{doi:10.1111/itor.12422} para generar la soluci\'on que ellos plantean.

\subsection{Modelo General}\label{model:general}
\noindent
\textbf{Par\'ametros}:\\
\begin{itemize}
    \item $G = (V,E)$ un grafo completo donde $V = \{1,...,n\}$ es un conjunto de v\'ertices y $E = \{(i,j)|i,j\in V\}$ conjunto de arcos que los unen.\\
    \item $S_{i}$ puntaje asociado al nodo $i \in N$ donde $S_{1} = S_{n} = 0$\\
    \item $C_{ij}$ costo asociado a recorrer el arco $(i,j) \in E$, que tambi\'en se puede referir a las distancia entre los nodos.\\
    \item $T_{max}$ es el tiempo o distancia m\'axima que puede demorar cada ruta.\\
    \item $m$ cantidad de participantes, que definen la cantidad de rutas.\\
\end{itemize}
\textbf{Variables}:\\
\begin{itemize}
    \item $y_{ik}\; (i = 1,...,n;\, k = 1,...,m)$ variable binaria que toma el valor de 1 si el v\'ertice $i$ es visitado en la ruta $k$, 0 sino.\\
    \item $x_{ijk}\; (i,j = 1,...,n;\, k = 1,...,m)$ variable binaria que toma el valor de 1 si el arco $(i,j)$ es visitado en la ruta $k$, 0 sino. Ya que $C_{ij} = C_{ji}$, s\'olo est\'a definido $x_{ijk}(i<j)$.\\
    \item $U$ es un subconjunto de $V$.\\
\end{itemize}
\textbf{Funci\'on Objetivo}:
\begin{equation}\label{eq:fx}
    Max \sum_{i=2}^{|N|-1}\sum_{k=1}^{m} S_{i}y_{ik}
\end{equation}
El objetivo de la funci\'on es maximizar la suma del puntaje obtenido entre las rutas.\\

\noindent
\textbf{Restricciones}:\\
\begin{equation}\label{eq:cons1}
    \sum_{j=2}^{n}\sum_{k=1}^{m}x_{1jk} = \sum_{i=1}^{n-1}\sum_{k=1}^{m}x_{ink} = m
\end{equation}
Esta restricci\'on obliga que toda ruta parta del nodo de inicio y llegue al nodo de llegada.
\begin{equation}\label{eq:cons2}
    \sum_{i<j}x_{ijk} + \sum_{i>j}x_{jik} = 2y_{jk}\; (j=2,...,n-1;\,k=1,...,m)
\end{equation}
Esta restricci\'on asegura la conectividad de cada ruta.
\begin{equation}\label{eq:cons3}
    \sum_{k=1}^{m}y_{ik} \leq 1\; (i=2,...,n-1)
\end{equation}
Esta restricci\'on asegura que cada v\'ertice, a excepci\'on del de inicio y de llegada, sea visitado a lo mas una vez.
\begin{equation}\label{eq:cons4}
    \sum_{i=1}^{n-1}\sum_{j>k}C_{ij}x_{ijk} \leq T_{max}\; (k = 1,...,m)
\end{equation}
Esta restricci\'on limita el tiempo o la distancia a recorrer de cada ruta.
\begin{equation}\label{eq:cons5}
    \sum_{i,j\in U}x_{ijk} \leq |U| - 1\; (U\subset V\setminus \{1,n\};\, 2\leq |U| \leq n-2;\, k=1,...,m)
\end{equation}
Esta restricci\'on se asegura que las subrutas est\'en prohibidos.
\begin{equation}\label{eq:nat1}
    x_{ijk} \in \{0,1\} \; (1\leq i < j \leq n;\, k = 1,...,m)
\end{equation}
\begin{equation}\label{eq:nat2}
    y_{1k} = y_{nk} = 1,\, y_{ik}\in \{0,1\}\; (i=2,...,n-1;\, k=1,...,m)
\end{equation}
La equaciones \eqref{eq:nat1} y \eqref{eq:nat2} aseguran la naturaleza de las variables del problema.\\
Se deduce que el espacio de b\'usqueda de este modelo debe ser del orden de $O(2^{n^3})$ ya que corresponder\'ia a una permutaci\'on binaria de una matriz tridimensional.

\subsection{Modelo de Bianchessi et al.}\label{model:bian}
\noindent
\textbf{Par\'ametros}:\\
\indent Los par\'ametros de este modelo son los mismos que los presentados en \ref{model:general}, por lo que se presentar\'an las variables, funci\'on objetivo y restricciones conforme a estos. S\'olo, y por comodidad se considera un par\'ametro como una especificaci\'on de otro, donde N son los v\'ertices intermedios del de inicio y el de llegada, es decir $N = V - \{v_1,v_n\}$\\

\noindent
\textbf{Variables}:\\
\indent Las variables binarias $x_{ij}$ para cada $(i,j) \in E$ e $y_{i}$ donde $i\in N$ toman valor 1 si el arco $(i,j)$ es usado y el nodo $i$ es visitado, 0 sino.\\
$z_{ij}\, (i,j)\in A\setminus \{1,n\}$ es una variable continua que representa la distancia que ha recorrido una ruta al momento de llegar a un v\'ertice $j$ viniendo de un v\'ertice $i$.\\
Para cada conjunto $S\subseteq N$, se definen $\delta^{+}(S) = \{(i,j) \in A:i\in S, j\notin S\}$ y $\delta^{-}(S) = \{(i,j) \in A:i\notin S, j\in S\}$  como los conjuntos de arcos saliendo y entrando el conjunto S respectivamente, con $\delta^{+}(i) = \delta^{+}(\{i\})$ y $\delta^{-}(i) = \delta^{-}(\{i\})$.\\

\noindent
\textbf{Funci\'on Objetivo}:\\
\begin{equation}\label{eq2:fx}
    Max \sum_{i\in N}S_{i}y_{i}
\end{equation}
Maximiza la suma de los puntajes de los nodos visitados.\\

\noindent
\textbf{Restricciones}:\\
\begin{equation}\label{eq2:cons1}
    \sum_{j\in N}x_{1j} = \sum_{i\in N}x_{i,n} = m
\end{equation}
Esta restricci\'on asegura que la cantidad de rutas que salgan, sean los mismos que lleguen y tienen que ser m.
\begin{equation}\label{eq2:cons2}
    \sum_{(j,i)\in \delta^{-}(i)}x_{ji} = \sum_{(i,j)\in \delta^{+}(i)}x_{ij} = y_{i} 
\end{equation}
Esta restricci\'on asegura que la cantidad de rutas que fluyan por el i-\'esimo nodo sea a los mas uno.
\begin{equation}\label{eq2:cons3}
    z_{0j} = C_{0j}x_{0j}
\end{equation}
Esta restricci\'on asegura que el flujo se origine del nodo inicial.
\begin{equation}\label{eq2:cons4}
    \sum_{(i,j)\in \delta^{+}(i)}z_{ij} - \sum_{(j,i)\in \delta^{-}(i)}z_{ji} = \sum_{(i,j)\in \delta^{+}(i)}C_{ij}x_{ij}
\end{equation}
Esta restricci\'on asegura la continuidad del flujo actualizando el valor de llegada a cada nodo.
\begin{equation}\label{eq2:cons5}
    z_{ij} \leq (T_{max} - C_{j,n})x_{ij}\; ((i,j)\in E\setminus\{1,n\})
\end{equation}
Esta restricci\'on asegura que el costo de cada flujo resultante no sea mayor al m\'aximo permitido.\\
\begin{equation}\label{eq2:cons6}
    z_{ij} \geq (C_{1i} + C_{ij})x_{ij}\; ((i,j)\in E\setminus\{1,n\})
\end{equation}
Esta restricci\'on impone un l\'imite inferior en los valores que tome la variable \textbf{z} de forma que se restrinjan los rango de los valores posibles que esta variable pueda tomar en soluciones intermedias. Notar que si $x_{ij} = 1$, entonces $z_{ij}$ denota la distancia de llegada al nodo j, mientras que $x_{ij} = 0 \implies z_{ij} = 0$.
\begin{equation}\label{eq2:cons7}
    y_{i} \in \{0,1\}\; (i\in N)
\end{equation}
\begin{equation}\label{eq2:cons8}
    x_{ij} \in \{0,1\}\; ((i,j)\in E\setminus\{1,n\})
\end{equation}
\begin{equation}\label{eq2:cons9}
    0 \leq x_{1n} \leq m
\end{equation}
\begin{equation}\label{eq2:cons10}
    \sum_{(i,j)\in A\setminus\{(1,n)\}}C_{ij}x_{ij}\leq m \times T_{max}
\end{equation}
Las ecuaciones \eqref{eq2:cons7}, \eqref{eq2:cons8} y \eqref{eq2:cons9} restrinjen la naturaleza de las variables, mientras que \eqref{eq2:cons10} es una restricci\'on hecha para fortalecer la formulaci\'on e impone una duraci\'on m\'axima global a las rutas.\\
El espacio de b\'usqueda de este modelo tiene un orden de $O(2^{n^2})$ ya que corresponde a la permutaci\'on de una matriz binaria.
\section{Representaci\'on}\label{represent}
La representaci\'on utilizada en el algoritmo propuesto para las soluciones en esta entrega se deriva principalemente de una matriz de adyasencia que indican los arcos que conforman cada una de las rutas, de la cual junto a otras variables se construyen $m$ listas con los nodos que conforman la ruta, una lista de tama\~no $m$ con el tiempo que se demora cada ruta y un puntaje asociado a la soluci\'on.

Esta soluci\'on se genera a partir de la extracci\'on de datos que entregan tres variables que van cambiando durante la ejecuci\'on del algoritmo, las cuales son una lista binaria de tama\~no $n$ que indica si el $i$-\'esimo vecino fue visitado o no, una matriz binaria $x$ de $n\times n$ que indica si el arco $(i,j)$ fue utilizado, y una matriz de valores continuos $z$ de tama\~no $n\times n$ que indica el tiempo de llegada de al nodo $i$ desde el nodo $j$.

El espacio de b\'usqueda viene dado a lo mas por la permutaci\'on de una matriz binaria de adyasencia de tama\~no $n \times n$, lo cual nos da un orden m\'aximo de $O(2^{n^2})$ pero que se ve disminuido dependiendo de la cantidad de rutas que tenga la isntanciaci\'on y dependiendo del tiempo m\'aximo de cada ruta, lo cual igualmente nos va a limitar la cantidad de soluciones factibles.

\section{Descripci\'on del algoritmo}\label{descript}
Debido a la naturaleza del problema, realizar BackTrack con GBJ era lo mismo que hacer un BackTrack completo, debido a que el grafo de restricciones era totalmente conexo, raz\'on por la cual la implementaci\'on que se presenta a continuaci\'on es la de un BackTracking completo y no existe una funci\'on de GBJ.\\

Una vez leidos todos los par\'ametros se genera una lista de v\'ertices, los cuales corresponden a clases con posici\'on x, posici\'on y, y un puntaje asociado, y una matriz de ejes de tama\~no $n \times n$ con la distancia euclidiana entre $(i,j)$ que es igual a $(j,i)$, que luego se pasan por referencia a una clase instancia del problema la cual posee referencias a todas las variables de forma de utilizarlas como variables globales al momento de ejecutar el algoritmo e intentar mantener el espacio de memoria utilizada por las variables constante.

El algoritmo de Backtrack se hace realizando una recursi\'on con paso de un par\'ametro entero sin declariaci\'on de variables dentro de el y con llamado a funciones para ser inmediatamente evaluadas, por lo cual la memoria stack utilizada por el algoritmo recursivo es a lo mas un \'arbol de profundidad $log(n)$ con a lo mas $m$ hijos por nodo ra\'iz relativo, lo cual genera una cantidad de hojas de $m^{log(n)}$.

El algoritmo recibe como par\'ametro un \'indice $i$ y eval\'ua como caso base si este $i$ corresponde al \'indice $n-1$ para saber si es que ya se recorrieron todos los $i$-\'esimos vecinos anteriores, luego llama a un chequeo para ver si del $n$-\'esimo nodo se pueden conectar los nodos que hasta el momento generan las rutas sin pasarse del tiempo m\'aximo permitido, en caso de poderse, se verifica si la soluci\'on actual es mejor que la existente y en caso de no ser as\'i simplemente se ignora. Luego se eliminan todos los cambios hechos en la matriz al intentar conectar el \'ultimo nodo con los de las rutas hasta el momento y se retorna al nivel anterior de recursi\'on. El segundo caso que el algoritmo verifica es si es que no existen las cantidades de rutas que la instancia pide, en cuyo caso intenta conectar el $i$-\'esimo nodo actual al nodo de origen, si puede lo conecta y llama a la recursi\'on para el siguiente \'indice y al volver deshace el cambio. Finalmente el tercer caso se verifica independiente de los otros dos, de forma que al volver al nivel recursivo del segundo caso habiendo sido exitoso, de todas formas se eval\'ua el tercer caso, utilizando as\'i el $i$-\'esimo nodo ya sea, sin necesidad de que hayan $m$ rutas, o habi\'endolas, eval\'uandolo de todas formas si es que no ha sido visitado, para esto se verifica si este nodo se puede conectar con los \'ultimos nodos que componen las rutas actuales, de forma de mantener la conectividad y la cantidad de rutas actuales, evitando as\'i que existan subrutas discontinuas o con mayor grado de salida que $1$. En caso de poder conectarse con alguno lo hace, para luego al volver deshacer los cambios, para luego igualmemente seguir con el siguiente \'indice sin haberse conectado.

A continuaci\'on se muestra un pseudo-c\'odigo de formageneral que explica el algoritmo recursivo:

\begin{algorithm}[H]
\KwData{routesAmm = 0, addRoutes = true}
\KwResult{Best route for instance}
begin \textbf{bTrack(i)}\;
\If{i is end node}{
\For{last nodes in routes}{
try to connect with end node\;
}
\If{connections possible}{
\eIf{this is best solution}{
update best solution\;}{
ignore it\;
}
}
}
\ElseIf{addRoutes AND i is not visited}{
\eIf{connection to initial node possible}{
connect\;
bTrack(i+1)\;
undoChanges\;
\tcc{Connecting and undoing updates routesAmm and addRoute}
}{
bTrack(i+1)
}
}
\If{i is not visited}{
\For{last nodes in routes}{
\If{connection possible}{
connect\;
bTrack(i+1)\;
undo\;
}
}
bTrack(i+1)
}
\tcc{Initial call is done with bTrack(1)}
\end{algorithm}


A nivel de implementaci\'on, las conexiones se hacen al verificarse, es decir, invoca a una funci\'on que retorna verdadero si es se pudo actualizar las matrices $x$ y $z$, al igual que el arreglo $y$ indicando que se agreg\'o el nodo, y falso si no se pudo. Ya que las conexiones y las verificaciones son en tiempo constante debido a que se acceden a \'indices espec\'ificos de arreglos fijos, no agrega mucha complejidad a la recursividad.

La funci\'on de recursividad de este algoritmo viene representada de forma muy general como:

\begin{align*}
    T(k) = m·T(k+1) + m\\
    T(n) = m\\
\end{align*}

Lo cual indica que aproximadamente el tiempo de ejecuci\'on del algoritmo es de $O(m^n)$ sin considerar el impacto que tiene el tiempo m\'aximo de cada ruta.
\section{Experimentos}\label{experiments}
Los experimentos se realizaron en un Notebook HP Pavilion 15-cx0003la, con procesador Intel Core i5 de 2,3 GHz, 8GB de RAM y en sistema operativo Ubuntu 18.04LTS.

Los par\'ametros se definieron seg\'un los trata la literatura, teniendo una cantidad $n$ de nodos en el problema, una cantidad $m$ de rutas, un tiempo $Tmax$ que restringe las rutas, y posteriormente la ubicaci\'on cartesiana de los puntos y su puntaje correspondiente. Como ya mencionado anteriormente, se almacenaron en un arreglo de vertices con puntaje asociado y una matriz de adyasencia cuyo valor $(i,j)$ es la distancia euclideana entre $i$ y $j$.

Ya que el tiempo de ejecuci\'on del algoritmo implementado igualmente es alto y hasta el \'ultimo momento de implementaci\'on del c\'odigo se obten\'ian algunos resultados erroneos, solamente se recolect\'o datos del set de instancias de 24 nodos que existen actualmente en la literatura, de los cuales se obtuvo un mejor puntaje seg\'un el algoritmo y el tiempo de ejecuci\'on que se demor\'o en obtenerlo. Otros sets igualmente fueron testeados, pero ya que por razones de tiempo no se pudieron probar todas las instancias de estos, no se recolectaron los datos.
\section{Resultados}\label{results}
Los resultados y sus comparaciones se muestran en el Cuadro \ref{tab:table1}, donde se evidencia el alto impacto que Tmax tiene sobre el tiempo de ejecuci\'on del algoritmo, ya que como es Backtrack, inmediatamente desecha las ramas donde se generan soluciones no v\'alidas, al aumentar Tmax, aumentan las soluciones v\'alidas y por lo tanto se recorre mas el espacio de b\'usqueda completo.

Adem\'as notamos como para las instancias p2.2.d, p2.2.e, p2.2.j, p2.2.k y p2.3.h no se obtuvieron rutas \'optimas, lo cual representa a un $85\%$ de efectividad con respecto de la soluci\'on mostrada en Boussier et al\cite{exactAlgorithm}, y obteniendo en un $61\%$ de los casos un tiempo de ejecuci\'on considerablemente peor seg\'un la notaci\'on y las sifras significativas utilizadas.

Esta diferencia en resultados de rutas \'optimas se debe a probablemente un error en la implementaci\'on, se utiliza hasta el \'ultimo momento para revisar el algoritmo pero hasta el momento no se ha logrado identificar donde recae el problema por el que se ignoran estas soluciones. En cuanto al tiempo de ejecuci\'on, es de esperarse que un Backtrack se demore esta cantidad de tiempo y aumente de esa forma cuando la cantidad de soluciones factibles aumenta, y aunque se revis\'o persistentemente la implementaci\'on buscando donde podr\'ia mejorarse el tiempo de ejecuci\'on, las mejoras posibles s\'olo implicaban una mejora en cantidad constante y poco significativa para la ejecuci\'on final del problema.
\begin{table}[H]
\begin{tabularx}{\textwidth}{lXcXlXcXcXcXc}
\hline
\multicolumn{1}{c}{\multirow{2}{*}{\textbf{Instance}}} & \multirow{2}{*}{\textbf{m}} & \multicolumn{1}{c}{\multirow{2}{*}{\textbf{Tmax}}} & \multicolumn{2}{c}{\textbf{Boussier 2007}} & \multicolumn{2}{c}{\textbf{BackTrack}} \\ \cline{4-7} 
\multicolumn{1}{c}{}                                   &                             & \multicolumn{1}{c}{}                               & \textbf{Score}      & \textbf{CPU(s)}      & \textbf{Score}    & \textbf{CPU(s)}    \\ \hline
p2.2.a                                                 & 2                           & 7.5                                                & 90                  & 0                    & 90                & 0                  \\
p2.2.b                                                 &                             & 10                                                 & 120                 & 0                    & 120               & 0                  \\
p2.2.c                                                 &                             & 11.5                                               & 140                 & 0                    & 140               & 0                  \\
p2.2.d                                                 &                             & 12.5                                               & 160                 & 0                    & 150               & 0                  \\
p2.2.e                                                 &                             & 13.5                                               & 190                 & 0                    & 170               & 1                  \\
p2.2.f                                                 &                             & 15                                                 & 200                 & 0                    & 200               & 2                  \\
p2.2.g                                                 &                             & 16                                                 & 200                 & 0                    & 200               & 2                  \\
p2.2.h                                                 &                             & 17.5                                               & 230                 & 0                    & 230               & 5                  \\
p2.2.i                                                 &                             & 19                                                 & 230                 & 0                    & 230               & 9                  \\
p2.2.j                                                 &                             & 20                                                 & 260                 & 1                    & 230               & 14                 \\
p2.2.k                                                 &                             & 22.5                                               & 284                 & 0                    & 240               & 34                 \\ \hline
p2.3.a                                                 & 3                           & 5                                                  & 70                  & 0                    & 70                & 0                  \\
p2.3.b                                                 &                             & 6.7                                                & 70                  & 0                    & 70                & 0                  \\
p2.3.c                                                 &                             & 7.7                                                & 105                 & 0                    & 105               & 0                  \\
p2.3.d                                                 &                             & 8.3                                                & 105                 & 0                    & 105               & 0                  \\
p2.3.e                                                 &                             & 9                                                  & 120                 & 0                    & 120               & 1                  \\
p2.3.f                                                 &                             & 10                                                 & 120                 & 0                    & 120               & 2                  \\
p2.3.g                                                 &                             & 10.7                                               & 145                 & 0                    & 145               & 4                  \\
p2.3.h                                                 &                             & 11.7                                               & 165                 & 0                    & 160               & 9                  \\
p2.3.i                                                 &                             & 12.7                                               & 200                 & 0                    & 200               & 24                 \\
p2.3.j                                                 &                             & 13.3                                               & 200                 & 0                    & 200               & 36                 \\
p2.3.k                                                 &                             & 15                                                 & 200                 & 0                    & 200               & 122                \\ \hline
p2.4.a                                                 & 4                           & 3.8                                                & 10                  & 0                    & 10                & 0                  \\
p2.4.b                                                 &                             & 5                                                  & 70                  & 0                    & 70                & 0                  \\
p2.4.c                                                 &                             & 5.8                                                & 70                  & 0                    & 70                & 0                  \\
p2.4.d                                                 &                             & 6.2                                                & 70                  & 0                    & 70                & 0                  \\
p2.4.e                                                 &                             & 6.8                                                & 70                  & 0                    & 70                & 0                  \\
p2.4.f                                                 &                             & 7.5                                                & 105                 & 0                    & 105               & 1                  \\
p2.4.g                                                 &                             & 8                                                  & 105                 & 0                    & 105               & 2                  \\
p2.4.h                                                 &                             & 8.8                                                & 120                 & 0                    & 120               & 4                  \\
p2.4.i                                                 &                             & 9.5                                                & 120                 & 0                    & 120               & 11                 \\
p2.4.j                                                 &                             & 10                                                 & 120                 & 0                    & 120               & 18                 \\
p2.4.k                                                 &                             & 11.2                                               & 180                 & 0                    & 180               & 78                 \\ \hline
\end{tabularx}
\scriptsize{\caption{Resultados comparados con los de Boussier et al.\cite{exactAlgorithm}}\label{tab:table1}}
\end{table}


\section{Conclusiones}\label{conclusions}
Respecto a aproximaciones anteriores, no tan solamente la metodolog\'ia utilizada juega un rol importante en la calidad de las soluciones y en el tiempo en que estas se obtienen, sino que adem\'as es relevante las representaciones que se utilizan para esto. Es as\'i como en Bianchessi et al.\cite{doi:10.1111/itor.12422} notamos importantes resultados nuevos en cuanto a benchmarking y a la obtensi\'on de nuevos \'optimos para considerar globales al cambiar el modelo matem\'atico del algoritmo exacto para que este funcione con variables de a lo sumo dos indices de dimensi\'on. Mientras que en Ke et al.\cite{KE2016155} notamos que en busquedas locales se logran mejores resultados al cambiar las operaciones internas del algoritmo al momento de elegir nuevas posibles soluciones para diversificar e intensificar, lo cual llev\'o a la creaci\'on de una nueva metaheur\'istica. Aunque la mayor\'ia de los papers visitados en este documento resuelven instancias del problema TOP, ya que as\'i fueron buscados los documentos, no todos logran obtener resultados favorables en cuantro a encontrar nuevos resultados globales o tiempos de ejecuci\'on notoriamente mas r\'apidos, en general muestran nuevas aproximaciones competitivas con respecto a los mejores, es tan s\'olo en los \'ultimos dos papers mencionados recientemente donde se logran resultados significativos en a\~nos, debido a, como ya se mencion\'o, la utilizaci\'on de nuevas representaciones, modelos, y metaheur\'isticas.

Respecto a este acercamiento, la elecci\'on de BackTrack con GBJ termin\'o llevando a la implementaci\'on de un BackTrack com\'un y corriente, debido a que como el problema posee un grafo de restricciones totalmente conexo, el salto al paso anterior parece tan bueno como cualquier otro para que cumpla con el requisito de ser un salto GBJ. Quiz\'as CBJ podr\'ia haber resultado mas eficiente, debido a que por la naturaleza de los caminos, al momento del algoritmo revisar las conexiones finales habr\'ia solamente deshecho los \'ultimos v\'ertices que visitan las rutas conflictivas, lo cual podr\'ia haber reducido tiempo de ejecuci\'on de manera considerable, aunque a medida que se aumenta el tiempo m\'aximo de cada ruta ambas heur\'isticas de salto habr\'ian convergido igualmente al mismo tiempo de ejecuci\'on.

Aunque el algoritmo propuesto entrega soluciones a la mayor\'ia de los casos de prueba existente en la literatura en tiempos que bordean las horas, no est\'a entregando resultados similares a los que de referencia debiese dar, es por esto que aunque se considera un buen acercamiento para obtener resultados globales, no est\'a bien logrado. A pesar de esto, se considera dentro de las posibles implementaciones de BackTrack para este problema, como una implementaci\'on competitiva y hasta mas eficiente que la media al ocupar depresentaciones de a lo mas dos \'indices, inspir\'andose en el modelo propuesto por Bianchessi et al.\cite{doi:10.1111/itor.12422}.

Para mejorar este algoritmo, una buena forma ser\'ia arregl\'andolo para que entregue las soluciones globales en las intancias que no est\'an correspondiendo, lo cual se deduce es debido a un problema puntual de implementaci\'on y no del algoritmo en si, ya que revisi\'on tras revisi\'on se ha verificado que la recursi\'on en su para recorrer el espacio de b\'usqueda est\'a bien hecho. Respecto al tiempo de ejecuci\'on, no es posible al momento de entregar este informe determinar alguna forma para mejorarlo.

\bibliographystyle{unsrt}
\bibliography{Referencias}

\end{document} 
